\section{PTS: Pure Type System}

\begin{definition}[PTS]
Un PTS $\mathcal{P} $ esta definido por $\{ \mathcal{S}, \mathcal{V}, \mathcal{A}, \mathcal{R}\}$ donde

\begin{itemize}
    \item{$\mathcal{S}$} es un conjunto sorts
    \item{$\mathcal{V}$} es un conjunto de variables
    \item{$\mathcal{A}$} es un conjunto no vacio de $\mathcal{S}\times\mathcal{S}$ llamados axiomas 
    \item{$\mathcal{R}$} es un conjunto de $\mathcal{S}\times\mathcal{S}\times\mathcal{S}$ llamadas $\Pi-Reglas$ 
\end{itemize}
\end{definition}

\begin{definition}[Pseudotérminos]
El conjunto T de pseudoterminos de un PTS $ \mathcal{P} = \{ \mathcal{S}, \mathcal{V}, \mathcal{A}, \mathcal{R}\}$ 
es el menor conjunto que satisface lo siguiente:
\begin{itemize}
    \item{} $\mathcal{S} \cup \mathcal{V} \subset \mathcal{T}$
    \item{} Si $a \in \mathcal{T}$ y $b \in \mathcal{T}$ entonces $ab \in\mathcal{T}$
    \item{} Si $A \in \mathcal{T}$,  $B \in \mathcal{T}$  y $x \in \mathcal{V}$  entonces $(\lambda x:A.B) \in\mathcal{T}$
    \item{} Si $A \in \mathcal{T}$,  $B \in \mathcal{T}$  y $x \in \mathcal{V}$  entonces $(\Pi x:A.B) \in\mathcal{T}$
\end{itemize}
\end{definition}

\begin{definition}[PTS funcional]
Un PTS se dice \emph{funcional} si
\begin{itemize}
\item $\langle s_1, s \rangle \in \mathcal{A}$ y $\langle s_1, s'\rangle \in \mathcal{A}$ implica $s = s'$;
\item $\langle s_1, s_2, s\rangle \in \mathcal{R}$ y $\langle s_1, s_2, s' \rangle \in \mathcal{R}$ implica $s = s'$.
\end{itemize}
\end{definition}
\begin{definition}[PTS full]
Un PTS se dice \emph{full} si para todo $s_1, s_2 \in \mathcal{S}$
existe un $s_3 \in \mathcal{S}$ con $\langle s_1, s_2, s_3 \rangle \in \mathcal{R}$.
\end{definition}


\begin{definition}[Relación $\vdash$]
Definimos a la relacion $\vdash$, con $\vdash \subseteq \mathcal{C}\times\mathcal{T}\times\mathcal{T}$ como
la menor relación que cumple:


\[
\begin{array}{llcr}
	(Srt) & &\infer{\emptyset \vdash s_{1} : s_{2}}{} & (s_{1},s_{2}) \in\mathcal{A} \\ \\
	(Var) & &\infer{\Gamma, x:A \vdash x:A}{\Gamma\vdash A:s} & \\ \\
	(Wk)  & &\infer{\Gamma, x:A \vdash b:B}{\Gamma\vdash b:B & \Gamma\vdash A:s} & b\in\mathcal{S}\cup\mathcal{V} \\ \\
	(Pi)  & &\infer{\Gamma \vdash \Pi x:A.B : s_{3}}{\Gamma\vdash A:s_{1} & \Gamma, x:A \vdash B:s_{2}} &  (s_1,s_2,s_3)\in\mathcal{R}\\ \\
	(Lda) & &\infer{\Gamma \vdash \lambda a:A.b : \Pi x:A.B }{\Gamma\vdash A:s_{1} & \Gamma, x:A \vdash b:B &\Gamma, x:A \vdash B:s_{2}} &  (s_1,s_2,s_3)\in\mathcal{R}\\ \\
	(App) & &\infer{\Gamma \vdash a b : A[x:= b] }{\Gamma\vdash a : \Pi x:B.A & \Gamma \vdash b:B} &  \\ \\
	(Cnv) & &\infer{\Gamma \vdash a:B}{\Gamma\vdash a:A & \Gamma\vdash B:s} & A \simeq B \\ \\
	
\end{array}
\]

\end{definition}

