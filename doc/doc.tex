\documentclass{article}
\usepackage[utf8]{inputenc}

\usepackage{amsthm}
\usepackage{amssymb}
\usepackage{proof}


\theoremstyle{definition}
\newtheorem{definition}{Definición}[section]
\newtheorem{theorem}{Teorema}[section]
\newtheorem{lemma}[theorem]{Lema}
\theoremstyle{remark}
\newtheorem*{remark}{Convencion}

\begin{document}

\title{Manual de Hue}
\author{Nicolas Cabral}

\maketitle

\begin{abstract}
The abstract text goes here.
\end{abstract}

\section{Hue}

  Es un asistente de pruebas basados en el Calculo de Construccions $\lambda C$


\section{PTS: Pure Type System}

\begin{definition}
Un PTS $\mathcal{P} $ esta definido por $\{ \mathcal{S}, \mathcal{V}, \mathcal{A}, \mathcal{R}\}$ donde

\begin{itemize}
    \item{$\mathcal{S}$} es un conjunto sorts
    \item{$\mathcal{V}$} es un conjunto de variables
    \item{$\mathcal{A}$} es un conjunto no vacio de $\mathcal{S}\times\mathcal{S}$ llamados axiomas 
    \item{$\mathcal{R}$} es un conjunto de $\mathcal{S}\times\mathcal{S}\times\mathcal{S}$ llamadas $\Pi-Reglas$ 
\end{itemize}
\end{definition}

\begin{definition}
El conjunto T de pseudoterminos de un PTS $ \mathcal{P} = \{ \mathcal{S}, \mathcal{V}, \mathcal{A}, \mathcal{R}\}$ 
es el menor conjunto que satisface lo siguiente:
\begin{itemize}
    \item{} $\mathcal{S} \cup \mathcal{V} \subset \mathcal{T}$
    \item{} Si $a \in \mathcal{T}$ y $b \in \mathcal{T}$ entonces $ab \in\mathcal{T}$
    \item{} Si $A \in \mathcal{T}$,  $B \in \mathcal{T}$  y $x \in \mathcal{V}$  entonces $(\lambda x:A.B) \in\mathcal{T}$
    \item{} Si $A \in \mathcal{T}$,  $B \in \mathcal{T}$  y $x \in \mathcal{V}$  entonces $(\Pi x:A.B) \in\mathcal{T}$
\end{itemize}

\end{definition}

\begin{definition}Definimos a la relacion $\vdash$, con $\vdash \subseteq \mathcal{C}\times\mathcal{T}\times\mathcal{T}$ como
la menor relación que cumple:

\begin{equation}
Srt\qquad\qquad\infer{\emptyset \vdash s_{1} : s_{2}}{}\qquad\qquad (s_{1},s_{2}) \in\mathcal{A}
\end{equation}
\end{definition}




\section{Teoría del calculo de $\lambda C$ }

\begin{definition}
$\lambda C$ es un PTS definido por $\{ \mathcal{S}, \mathcal{V}, \mathcal{A}, \mathcal{R}\}$ tales que:

\begin{itemize}
    \item{} $\mathcal{S} = \{Prop,Type\}$
    \item{} $\mathcal{V} = \dots $
    \item{} $\mathcal{A} = \{(Prop, Type)\}$
    \item{} $\mathcal{R} = \{(Prop, Prop, Prop), (Type, Prop, Prop), (Type, Type, Type)\}$
\end{itemize}

\end{definition}

\begin{lemma}
$\lambda C$ es un PTS \it{full}
\end{lemma}
\begin{proof}
\end{proof}


\subsection{Terminos}
\subsection{Reduccion}
\subsection{Contextos}
\subsection{Typechecker}
~\cite{Jutting93checkingalgorithms}
~\cite{Benthem:93}
~\cite{DBLP:conf/types/JuttingMP93}

\begin{equation}
    \label{simple_equation}
\infer[\mathrm{Srtnsd}]
{\vdash s_{1} : s_{2}}{}
\end{equation}

\begin{equation}
    \label{simple_equation}
\infer[\mathrm{Varnsd}]
{\Gamma, x:A \vdash_{nsd} x : A}{\Gamma \vdash A :\twoheadrightarrow s}
\end{equation}

\begin{equation}
    \label{simple_equation}
\infer[\mathrm{Wknsd}]
{\Gamma, x:A \vdash_{nsd} b : B}{\Gamma \vdash_{nsd} b : B & \Gamma \vdash A :\twoheadrightarrow s}
\end{equation}

\begin{equation}
    \label{simple_equation}
\infer[\mathrm{Pinsd}]
{\Gamma \vdash_{nsd} \Pi x:A.B :s_{3}}{\Gamma \vdash A :\twoheadrightarrow s_{1} & \Gamma, x:A \vdash_{nsd} B :\twoheadrightarrow s_{2}}
\end{equation}

\begin{equation}
    \label{simple_equation}
\infer[\mathrm{Ldansd}]
{\Gamma \vdash_{nsd} \lambda x:A.b :\Pi x:A.B}{\Gamma \vdash A :\twoheadrightarrow s_{1} & \Gamma, x:A \vdash_{nsd} b:\twoheadrightarrow B & \Gamma, x:A \vdash_{nsd} B :\twoheadrightarrow s_{2}}
\end{equation}

\begin{equation}
    \label{simple_equation}
\infer[\mathrm{Appnsd}]
{\Gamma \vdash_{nsd} a\ b : A[x:=b]}{\Gamma \vdash a :\twoheadrightarrow \Pi x:A.B & \Gamma \vdash_{nsd} b:\twoheadrightarrow B }
\end{equation}


\section{Conclusion}
Write your conclusion here.


\bibliography{doc}{}
\bibliographystyle{plain}
\end{document}
