\section{Hue: Typechecker}
\begin{definition}
Los terminos del $\lambda$-calculo esta definidos por:

\begin{verbatim}
data CoCT =  IdT Int
           | VarT String
           | LamT CoCT CoCT
           | AppT CoCT CoCT
           | PiT CoCT CoCT
           | PropT
           | TypeT deriving (Eq)

type Type = CoCT
type Term = CoCT

\end{verbatim}

\end{definition}

\begin{definition}
Los contextos en Hue estan definidos de la siguiente manera:

\begin{verbatim}
type Context = [ (String, (Maybe Term, Type)) ]
\end{verbatim}

\end{definition}

\begin{definition}
Definimos $whnf$ como el operador que reduce un termino a su forma normal a la cabeza.

\begin{verbatim}
whnf :: Context -> CoCT -> NTerm
whnf c (IdT n)    = NAtom (AId n)
whnf c (VarT s)   = maybe (NAtom (AVar s)) (whnf c) (getDef c s)
whnf c (PiT s t)  = NAtom (APi s t)
whnf c PropT      = NAtom (AProp)
whnf c TypeT      = NAtom (AType)
whnf c (LamT s t) = NLam s t
whnf c (AppT m n) = case whnf c m of
            NLam _ t -> whnf c (app t n)
            NAtom t  -> NAtom (AApp t n)

\end{verbatim}
\end{definition}

\begin{definition}
Definimos la relacion $A \simeq_{\delta\beta} B$ de la siguiente manera

\begin{verbatim}
conv :: Context -> CoCT -> CoCT -> Bool
conv c s t = convT c (whnf c s) (whnf c t)
\end{verbatim}

\end{definition}

\begin{definition}
Definimos el operador $\twoheadrightarrow^{*} \Pi x:A.B$ de la siguiente manera

\begin{verbatim}
getPi :: Context -> CoCT -> Maybe (CoCT,CoCT)
getPi c t = case whnf c t of
              NAtom (APi m n) -> Just (m, n)
              _       -> Nothing
\end{verbatim}

Definimos el operador $\twoheadrightarrow^{*} s$ donde $s\in\cS$ de la siguiente manera

\begin{verbatim}
getSort :: Context -> CoCT -> Maybe CoCT
getSort c t = case whnf c t of
                NAtom AProp -> Just PropT
                NAtom AType -> Just TypeT
                _           -> Nothing

\end{verbatim}
\end{definition}



\begin{definition}[Relación $\vdash_{hue}$]
Definimos a la relacion $\vdash_{hue}$, con $\vdash_{hue} \subseteq \mathcal{C}\times\cT\times\cT$ como
la menor relación que cumple:


\[
\begin{array}{llcr}
	(Srt-hue) & &\infer{\Gamma \vdash_{hue} s_{1} : s_{2}}{} & (s_{1},s_{2}) \in\cA \\ \\ 
	(Var-hue) & &\infer{\Gamma \vdash_{hue} x : A}{} & x:A \in\Gamma \\ \\

	(Pi-hue)  & &\infer{\Gamma \vdash_{hue} \Pi x:A.B : s_{3}}{\Gamma\vdash_{hue} A:\twoheadrightarrow^{*} s_{1} & \Gamma, x:A \vdash_{hue} B:\twoheadrightarrow^{*}s_{2}} &  (s_1,s_2,s_3)\in\cR\\ \\
               & &                                             & x \not\in dom(\Gamma)\\
	(Lda-hue) & &\infer{\Gamma \vdash_{vtyp} \lambda a:A.b : \Pi x:A.B }{\Gamma\vdash_{vtyp} A:\twoheadrightarrow^{*} s_{1} & \Gamma, x:A \vdash_{vtyp} b: B
} & B \not\in\cS \lor \cA(B) \\
	(App-hue) & &\infer{\Gamma \vdash_{hue} a b : A[x:= b] }{\Gamma\vdash_{hue} a :\twoheadrightarrow^{*} \Pi x:B.A & \Gamma \vdash_{hue} b:B'} &  B\simeq B' \\ \\
	
\end{array}
\]

\end{definition}

\begin{lemma}{Equivalencia de $\vdash$ y $\vdash_{hue}$}
Sea $\vdash$ la relación del PTS asociado a $\lambda C$  entonces:
\begin{equation}
\Gamma \vdash a : A \Leftrightarrow \Gamma \vdash_{vctx} \land \Gamma \vdash_{hue} a:A
\end{equation}
\end{lemma}

\begin{proof}
chan!
\end{proof}

